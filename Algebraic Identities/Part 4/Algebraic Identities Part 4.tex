\documentclass{article}
\usepackage[utf8]{inputenc}
\usepackage{enumitem}
\usepackage{amsmath,amsthm,amssymb}
\usepackage[english]{babel}
\usepackage[a4paper, portrait, margin=1.2in]{geometry}
% \usepackage{ntheorem}
% \usepackage{theoremref}
\usepackage{xifthen}
\usepackage{relsize}
\usepackage[svgnames]{xcolor} % You can find CSS colours here https://www.w3schools.com/cssref/css_colors.asp

\title{Algebraic Identities Part 4}
\author{Rasmus Söderhielm}
\date{December 2021}

% \parindent 20pt
% \parskip 1em

\definecolor{MainColor}{hsb}{0, 0.8, 0.65}
\definecolor{SubColor}{hsb}{0.01, 0.90, 0.45}
\definecolor{SubSubColor}{hsb}{0.01, 0.90, 0.2}
% \colorlet{SubColor}{Teal}

% switch implementation from https://tex.stackexchange.com/questions/64131/implementing-switch-cases
\newcommand{\ifequals}[3]{\ifthenelse{\equal{#1}{#2}}{#3}{}}
\newcommand{\case}[2]{#1 #2} % Dummy, so \renewcommand has something to overwrite...
\newenvironment{switch}[1]{\renewcommand{\case}{\ifequals{#1}}}{}

\newcommand{\hint}{\\\textcolor{SubColor}{{H}{\relsize{-1}INT:}}\ }



\renewcommand*{\thesubsection}{\textcolor{MainColor}{\Alph{subsection}}}
\newcommand{\solution}{\subsubsection*{\textcolor{MainColor}{Solution}}}

\newcounter{theoremcounter}
% \newcounter{corollary}

\newtheoremstyle{maintheorem}% name of the style to be used
	{\topsep}% measure of space to leave above the theorem. E.g.: 3pt
	{\topsep}% measure of space to leave below the theorem. E.g.: 3pt
	{\itshape}% name of font to use in the body of the theorem
	{0pt}% measure of space to indent
	{\color{SubColor}\bfseries}% name of head font
	{.}% punctuation between head and body
	{ }% space after theorem head; " " = normal interword space
	{\thmname{#1}\thmnumber{ #2}\textnormal{\thmnote{ (#3)}}}
  
% \newtheoremstyle{mainsolution}% name of the style to be used
% 	{\topsep}% measure of space to leave above the theorem. E.g.: 3pt
% 	{\topsep}% measure of space to leave below the theorem. E.g.: 3pt
% 	{\normalfont}% name of font to use in the body of the theorem
% 	{0pt}% measure of space to indent
% 	{\color{SubColor}\bfseries}% name of head font
% 	{}% punctuation between head and body
% 	{ }% space after theorem head; " " = normal interword space
% 	{\thmname{#1}\thmnumber{ #2}\textnormal{\thmnote{ (#3)}}}

\theoremstyle{maintheorem}
\newtheorem{theorem}[theoremcounter]{\textcolor{SubColor}{Theorem}}
\newtheorem{corollary}{\textcolor{SubColor}{Corollary}}


\newcommand{\thmref}[1]{\textcolor{SubSubColor}{\textbf{Theorem \ref{#1}}}}
\newcommand{\corref}[1]{\textcolor{SubSubColor}{\textbf{Corollary \ref{#1}}}}
\renewcommand{\eqref}[1]{\textcolor{SubSubColor}{\textbf{Equation \ref{#1}}}}

% \theoremstyle{mainsolution}
% \newtheorem*{solution}{Solution}



\newcommand{\size}[2]{
	\begin{switch}{#1}
		\case{1}{#2}
		\case{2}{\bigl#2}
		\case{3}{\Bigl#2}
		\case{4}{\biggl#2}
		\case{5}{\Biggl#2}
	\end{switch}
}

\setlength{\parskip}{0.8em}

\begin{document}

\linespread{1.5}\selectfont

\maketitle

\section*{\color{MainColor}Problems} \label{section}

\subsection{
	\normalfont
	Let $a, b$ and $c$ be real numbers where $a \neq b$. If $c^3 = a^3 + b^3 + 3abc$, prove that $c=a+b$.
}

\solution

During the last lecture we learned about the following theorem and it's corollary.
\begin{theorem}\label{thm1}
	Let $a, b$ and $c$ be real numbers.
	\[ a^3 + b^3 + c^3 - 3abc = (a + b + c)(a^2 + b^2 + c^2 - ab - bc - ca) \]
\end{theorem}
\begin{corollary}\label{cor1}
	Let $a, b$ and $c$ be real numbers.

	If $a^3 + b^3 + c^3 = 3abc$, then either $a = b = c$ or $a + b + c = 0$ is true.
\end{corollary}

We start by rearrange the equation $ c^3 = a^3 + b^3 + 3abc $ into the following form:
\[ -a^3 - b^3 + c^3 = 3abc \]
This equation is now in the form of \corref{cor1}. Therefore we can conclude that either $-a = -b = -c$ or $-a - b + c = 0$ is true.

We know that $ -a = -b = -c $ is false since $a \neq b$. This leaves us with $-a - b + c = 0$, which we can easily rearrange into the form of $c = a + b$, thus proving the problem.

\subsection{
	\normalfont
	Let $x$, $y$ and $z$ be nonzero real numbers such that
	\[x + y + z = a,\ \frac{1}{x} + \frac{1}{y} + \frac{1}{z} = \frac{1}{a}\]
	Show that at least one of $x$, $y$, $z$ is equal to $a$.
}

\solution

We begin by recognizing that either $a - x$, $a - y$ or $a - z$ is equal to zero. Thus we can rephrase the problem into proving that $ (a - x)(a - y)(a - z) = 0 $.

Using the theorem we learned during the lesson we can expand the following expression.
\[(a-x)(a-y)(a-z) = a^3 - (x + y + z)a^2 + (xy + yz + zx)a - xyz\]
Substituting $a$ for $ x + y + z $, the expression $a^3 - a^3$ cancels out.
\[(a-x)(a-y)(a-z) = (xy + yz + zx)a - xyz\]
We divide $xy + yz + zx$ by $xyz$ and multiply it by $xyz$ to cancel out the division.
\[(a-x)(a-y)(a-z) = xyz\left(\frac{1}{x} + \frac{1}{y} + \frac{1}{z}\right)a - xyz\]
Now we can substitute $ \frac{1}{a} $ for $ \frac{1}{x} + \frac{1}{y} + \frac{1}{z} $ we get $ xyz\frac{a}{a} - xyz $. The fraction $ \frac{a}{a} $ cancels out, leading to $xyz - xyz$ also cancelling out.
This leaves us with
\[(a-x)(a-y)(a-z) = 0\]
This proves that either $a - x$, $a - y$ or $a - z$ is equal to zero and therefore either $a$, $b$ or $c$ is equal to $a$.

\subsection{
	\normalfont
	Solve the following system:
	\begin{align*}
		x + y + z       & = 10   \\
		x^2 + y^2 + z^2 & = 100  \\
		x^3 + y^3 + z^3 & = 1000
	\end{align*}
}

\solution

During a past lecture we learned about the following theorem:
\begin{theorem}\label{thm2}
	Let $a, b$ and $c$ be real numbers.
	\[ (a + b + c)^3 = a^3 + b^3 + c^3 + 3(a + b)(b + c)(c + a) \]
\end{theorem}

Let's begin by writing out \thmref{thm2}.
\[ (a + b + c)^3 = a^3 + b^3 + c^3 + 3(a + b)(b + c)(c + a) \]
We can substitute $10$ and $1000$ for $x + y + z$ and $x^3 + y^3 + z^3$ respectively. This leaves us with $1000 = 1000 + 3(a + b)(b + c)(c + a)$,
which we can simplify to $0 = (a + b)(b + c)(c + a)$.

Therefore we conclude that either $x + y = 0$ or $y + z = 0$ or $z + x = 0$ is true.
Since the sum of one of the pairs of variables is equal to $0$, we conclude that the third variable is equal to $10$ using the fact that $x + y + z = 10$.

To find the values of the other pair of variables let's temporarily assume that $z$ is equal to $10$. Inserting this value into the equation $x^2 + y^2 + z^2 = 100$, it then simplifies to $x^2 + y^2 = 0$.
Because the exponents in the terms $x^2$ and $y^2$ are both even, they are also nonnegative. Since they are nonnegative and add up to $0$, they both need to be equal to $0$.

From this we can conclude that there are three different solutions to the original system:
\begin{align*}
	x & = 10 &  &           & y & = 10 &  &           & z & = 10 \\
	y & = 0  &  & \text{or} & z & = 10 &  & \text{or} & x & = 0  \\
	z & = 0  &  &           & x & = 10 &  &           & y & = 0
\end{align*}

\subsection{
	\normalfont
	Let $a, b$ and $c$ be real numbers such that $a + b + c = 0$. Show that $2(a^5 + b^5 + c^5) = 5abc(a^2 + b^2 + c^2)$.
}

\solution

Since $a + b + c = 0$, we can derive the following equation using Theorem \thmref{thm1}, mentioned in Problem A.
\begin{equation}\label{eq1}
	abc = \frac{1}{3}(a^3 + b^3 + c^3)
\end{equation}

Let's consider the expression $5abc(a^2 + b^2 + c^2)$.
Using \eqref{eq1} we substitute $\frac{1}{3}(a^3 + b^3 + c^3)$ for $abc$, resulting in the expression $\frac{5}{3}(a^3 + b^3 + c^3)(a^2 + b^2 + c^2)$.
After expanding and factorising we are left with the following expression.
\[ \frac{5}{3}\bigl(a^5 + b^5 + c^5 + a^2b^2(a + b) + b^2c^2(b + c) + c^2a^2(c + a)\bigr) \]

Since $a + b + c = 0$, we can derive the following three equations.
\begin{align*}
	-a & = b + c \\
	-b & = c + a \\
	-c & = a + b
\end{align*}
Using these equations we substitute $-a, -b$ and $-c$ into our expression to get the expression $\frac{5}{3}(a^5 + b^5 + c^5 - a^2b^2c - ab^2c^2 - a^2bc^2)$.
After factorising $abc$ we are left with the following result.
\[ \frac{5}{3}\bigl(a^5 + b^5 + c^5 - abc(ab + bc + ca)\bigr) \]

We learned about the following theorem during a past lecture.
\begin{theorem}\label{thm3}
	Let $a, b$ and $c$ be real numbers.
	\[ (a + b + c)^2 = a^2 + b^2 + c^2 + 2(ab + bc + ca) \]
\end{theorem}
Consider the equation $a + b + c = 0$. This means that the equation $(a + b + c)^2 = 0$ is also true.
Using \thmref{thm3} we can modify the equation to get the equation $a^2 + b^2 + c^2 + 2(ab + bc + ca) = 0$, which after rearranging gives us the following result.
\[ ab + bc + ca = -\frac{1}{2}\size2(a^2 + b^2 + c^2\size2) \]

Returning to the previous expression we can now substitute $-\frac{1}{2}\size2(a^2 + b^2 + c^2\size2)$ for $ab + bc + ca$,
giving us the following result.
\[ \frac{5}{3}\size3(a^5 + b^5 + c^5 + \frac{1}{2}abc\size2(a^2 + b^2 + c^2\size2)\size3) \]

As we derived this expression from the previous expression $5abc\size2(a^2 + b^2 + c^2\size2)$, the following equation holds true.
\begin{equation*}
	5abc\size2(a^2 + b^2 + c^2\size2) = \frac{5}{3}\size2(a^5 + b^5 + c^5\size2) + \frac{5}{6}abc\size2(a^2 + b^2 + c^2\size2)
\end{equation*}

Multiplying the preceding equation by $\frac{6}{5}$ and then simplifying it gives us the resulting equation $2(a^5 + b^5 + c^5) = 5abc(a^2 + b^2 + c^2)$,
which is the equation we set out to prove.


\subsection{
	\normalfont
	Let $a, b$ and $c$ be nonzero real numbers such that $a + b + c = 0$ and $a^3 + b^3 + c^3 = a^5 + b^5 + c^5$.
	Determine the exact value of $a^2 + b^2 + c^2$.
	\hint Expand $(a^2 + b^2 + c^2)(a^3 + b^3 + c^3)$
}

\solution

Let's do as the problem asks and begin expanding the expression $(a^2 + b^2 + c^2)(a^3 + b^3 + c^3)$.
This results in the following expression.
\[ a^5 + b^5 + c^5 + a^3b^2 + a^2b^3 + b^3c^2 + b^2c^3 + c^3a^2 + c^2a^3 \]
After substituting $a^3 + b^3 + c^3$ for $a^5 + b^5 + c^5$ and factorising we are left with the following expression.
\[ a^3 + b^3 + c^3 + a^2b^2(a + b) + b^2c^2(b + c) + c^2a^2(c + a) \]

In the same way as we did in question D, we can rearrange $a + b + c = 0$ into $-a = b + c$, $-b = c + a$ and $-c = a + b$.

Substituting this into our previous equation and factorising out $abc$, we get the following result, which we will pair up with our starting expression.
\[ (a^2 + b^2 + c^2)(a^3 + b^3 + c^3) = a^3 + b^3 + c^3 - abc(ab + bc + ca) \]
Using \corref{cor1} we will substitute $3abc$ for $a^3 + b^3 + c^3$ and then divide both sides by 3abc,
giving us the following equation.
\begin{equation}\label{eq2}
	a^2 + b^2 + c^2 = 1 - \frac{ab + bc + ca}{3}
\end{equation}

To continue we first have to find $ab + bc + ca$.
But we're going to need to introduce the following theorem first.
\begin{theorem}\label{thm4}
	Let $a, b$ and $x$ be real numbers.
	\[ (x + a)(x + b) = x^2 + (a + b)x + ab \]
\end{theorem}
Using the equations established previously, we replace $a$, $b$ and $c$ with $-b-c$, $-c-a$ and $-a-b$ respectively,
giving us the following equation after expanding it.
\[ ab + bc + ca = a^2 + b^2 + c^2 + a(b + c) + b(a + c) + c(a + b) + ab + bc + ca \]
That after simplifying gives us this equation.
\begin{equation}\label{eq3}
	ab + bc + ca = -\frac{a^2 + b^2 + c^2}{2}
\end{equation}

We can now use \eqref{eq3} to replace $ab + bc + ca$ in \eqref{eq2}, giving us the resulting equation $ a^2 + b^2 + c^2 = 1 + \frac{a^2 + b^2 + c^2}{6}$.
After rearranging this equation we are left with the following equation.
\[ a^2 + b^2 + c^2 = \frac{6}{5} \]
Which is the answer we set out to find.

\subsection{
	\normalfont
	Solve the following equation.
	\normalsize
	\[ (x + 1)^{63} + (x + 1)^{62}(x - a) + (x + 1)^{61}(x - a)^2 + \cdots + (x + 1)^2(x - 1)^{61} + (x + 1)(x - 1)^{62} + (x - 1)^{63} = 0 \]
}

During the last lecture we learned about the following extension of the conjugate rule.
\begin{theorem}\label{conjExt}
	\[ a^n - b^n = (a - b)\sum_{k=1}^{n}a^{n-k}b^{k-1} \]
	% \hint Multiply both sides of the equation by $2 = (x + 1) - (x - 1)$.
\end{theorem}

We begin by recognizing that the left hand side of our equation is almost the same as the right hand side of \thmref{conjExt},
albeit with a missing multiplication by $a - b$.
The expression $x + 1$ and $x - 1$ would take the place of $a$ and $b$ respectively.

As stated, the only difference is that the expression from \thmref{conjExt} is multiplied by $a - b$,
which we can easily fix by multiplying both sides of our equation by $(x + 1) - (x - 1)$.
Since the right hand side of our equation is equal to $0$, it stays the same, giving us the following equation.
\[ \size2((x + 1) - (x - 1)\size2)\size3((x + 1)^{63} + (x + 1)^{62}(x - a) + (x + 1)^{61}(x - a)^2 + \cdots + (x + 1)(x - 1)^{62} + (x - 1)^{63}\size3) = 0 \]
Now that our equation is exactly the same as the one in \thmref{conjExt}, we can rewrite our equation into the following, more manageable, form.
\[ (x + 1)^{64} - (x - 1)^{64} = 0 \]

After moving $(x - 1)^{64}$ to the right hand side and taking the 64th root of both side we are left with two possibilities,
either $x + 1 = x - 1$ or $x + 1 = -(x - 1)$. We can conclude that the first possibility has to be false since after rearraigning it we are left with the contradiction $1 = -1$, therefore possibility two is the only possibility.
The second equation has the solution of $x = 0$, thusly giving us the value asked of us in the problem.

\end{document}