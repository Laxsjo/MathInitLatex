\documentclass{article}
\usepackage[utf8]{inputenc}
\usepackage{enumitem}
\usepackage{amsmath,amsthm,amssymb}
\usepackage[english]{babel}

\title{Algebraic Identities Part 4}
\author{Rasmus Söderhielm}
\date{December 2021}

% \parindent 20pt
% \parskip 1em


\newcommand{\janktitle}[1]{
	\textbf{#1:}\\
}



\renewcommand*{\thesubsection}{\Alph{subsection}}

\newtheorem{theorem}{Theorem}

\setlength{\parskip}{0.8em}

\begin{document}

\linespread{1.5}\selectfont

\maketitle

\section{Problems}


%\begin{enumerate}[label=\alph*.]
%    \item something
%    \item else
%\end{enumerate}

% \begin{itemize}
% 	\item[A] hello

% \end{itemize}
\subsection{
	\normalfont
	Let $a, b, c \in \mathbb{R} $ where $a \neq b$. If $c^3 = b^3 + b^3 + 3abc$, prove that $c=a+b$.
}

\begin{theorem}{lem}{little lemma} This is a small lemma. \end{theorem}
It follows from \ref{little lemma} that we have
\begin{theorem*}{thm} The main result. \end{theorem*}

\begin{theorem}
	Let $ a, b, c \in \mathbb{R} $ If $ a^3 + b^3 + c^3 = 3abc \Leftrightarrow a = b = c $ or $ a + b + c = 0 $
\end{theorem}

\begin{align*}
	b^3 + b^3 + 3abc & = c^3   \\
	b^3 + b^3 - c^3  & = -3abc \\
	-a^3 - b^3 + c^3 & = 3abc
\end{align*}
Using theorem 1 we can conclude
\begin{align*}
	-a = -b = c                 &  & or &  & -a - b + c & = 0     \\
	\text{which is false since} &  &    &  & c          & = a + b \\
	a \neq b
\end{align*}

\subsection{
	\normalfont
	Let $x$, $y$ and $z$ be non-zero real numbers such that
	\[x + y + z = a,\ \frac{1}{x} + \frac{1}{y} + \frac{1}{z} = \frac{1}{a}\]
	Show that at least one of $x$, $y$, $z$ is equal to $a$.
}

We begin by recognizing that either $a - x$, $a - y$ or $a - z$ is equal to zero. Thus we can rephrase the problem into proving that $ (a - x)(a - y)(a - z) = 0 $.

Using the theorem we learned during the lesson we can expand the following expression.
\[(a-x)(a-y)(a-z) = a^3 - (x + y + z)a^2 + (xy + yz + zx)a - xyz\]
Substituting $a$ for $ x + y + z $, the expression $a^3 - a^3$ cancels out.
\[(a-x)(a-y)(a-z) = (xy + yz + zx)a - xyz\]
We divide $xy + yz + zx$ by $xyz$ and multiply it by $xyz$ to cancel out the division.
\[(a-x)(a-y)(a-z) = xyz\left(\frac{1}{x} + \frac{1}{y} + \frac{1}{z}\right)a - xyz\]
Now we can substitute $ \frac{1}{a} $ for $ \frac{1}{x} + \frac{1}{y} + \frac{1}{z} $ we get $ xyz\frac{a}{a} - xyz $. The fraction $ \frac{a}{a} $ cancels out, leading to $xyz - xyz$ also cancelling out.
This leaves us with
\[(a-x)(a-y)(a-z) = 0\]
This proves that either $a - x$, $a - y$ or $a - z$ is equal to zero and therefore either $a$, $b$ or $c$ is equal to $a$.

\subsection{
	\normalfont
	Solve the following system:
	\begin{align*}
		x + y + z       & = 10   \\
		x^2 + y^2 + z^2 & = 100  \\
		x^3 + y^3 + z^3 & = 1000
	\end{align*}
}

\end{document}

