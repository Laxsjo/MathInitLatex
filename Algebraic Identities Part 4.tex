\documentclass{article}
\usepackage[utf8]{inputenc}
\usepackage{enumitem}
\usepackage{amsmath,amsthm,amssymb}
\usepackage[english]{babel}

\title{Algebraic Identities Part 4}
\author{Rasmus Söderhielm}
\date{November 2021}

% \parindent 20pt
% \parskip 1em


\newcommand{\janktitle}[1]{
	\textbf{#1:}\\
}

\renewcommand*{\thesubsection}{\Alph{subsection}}

\newtheorem{theorem}{Theorem}

\begin{document}

\maketitle

\section{Problems}

%\begin{enumerate}[label=\alph*.]
%    \item something
%    \item else
%\end{enumerate}

% \begin{itemize}
% 	\item[A] hello

% \end{itemize}
\subsection{
	\normalfont
	Let $a, b, c \in \mathbb{R} $ where $a \neq b$. If $c^3 = b^3 + b^3 + 3abc$, prove that $c=a+b$.
}

\begin{proof}
	\begin{theorem}
		Let $ a, b, c \in \mathbb{R} $ If $ a^3 + b^3 + c^3 = 3abc \Leftrightarrow a = b = c $ or $ a + b + c = 0 $
	\end{theorem}

	\begin{align*}
		b^3 + b^3 + 3abc & = c^3   \\
		b^3 + b^3 - c^3  & = -3abc \\
		-a^3 - b^3 + c^3 & = 3abc
	\end{align*}
	Using theorem 1 we can conclude
	\begin{align*}
		-a = -b = c                 &  & or &  & -a - b + c & = 0     \\
		\text{which is false since} &  &    &  & c          & = a + b \\
		a \neq b
	\end{align*}
\end{proof}





\end{document}

